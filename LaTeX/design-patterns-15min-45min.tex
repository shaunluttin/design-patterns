% $Header$

\documentclass{beamer}

\usepackage[english]{babel}
\usepackage[latin1]{inputenc}
\usepackage{times}
\usepackage{listings}

\title{Design Patterns}
\subtitle{aka Object Oriented Programming}

\AtBeginSection{\frame{\sectionpage}}
\AtBeginSubsection{\frame{\subsectionpage}}

\setbeamersize{text margin left=2mm,text margin right=2mm} 

\lstset{
    % https://tex.stackexchange.com/a/34690/137042
    basicstyle=\fontfamily{pcr}\footnotesize\color{yellow},
    backgroundcolor=\color{black},
    frame=single
}

\begin{document}

\begin{frame}
  \titlepage
\end{frame}

\begin{frame}{Outline}
  \tableofcontents
\end{frame}

\section{Fundamentals}

\begin{frame}{Abstraction}
    \par Delineates a simplified, context-specific representation of a thing.
    \begin{itemize}
        \item Ignores contextually irrelevant details.
        \item Includes contextually relevant details.
    \end{itemize}
\end{frame}

\begin{frame}{Encapsulation}
    \par Restricts outside access to a things parts.
    \par Bundles a things state with the routines that use that state.
\end{frame}

\begin{frame}{Inheritance}
    \par Grants one thing the capabilities of another thing.
    \par Is not the same as though often agrees with subtyping.
    \par Includes prototypal and class-based inheritance.
\end{frame}

\begin{frame}{Polymorphism}
    \par Pluralizes the numbers of types on which a routine can operate.
    \begin{itemize}
        \item Ad hoc / static polymorphism (aka method overloading)
        \item Parameteric polymorphism (aka generics)
        \item Subtype polymorphism
    \end{itemize}
\end{frame}

\section{Principles}

\subsection{A class should have only one reason to change.}

\begin{frame}{Foobar}
    \par A reason to change is a responsibility or an axis-of-change.
    \par Changes refer to changes in source code not to variables at runtime.
    \par Example responsibilities: print an invoice, calculate tax.
    \par Loosely coupled responsibilities bring benefits:
    \begin{itemize}
        \item Clients can consume individual responsibilities.
        \item Responsibilities can change without breaking each other.
        \item Responsibilities can be recompiled independently.
    \end{itemize}
    \par The art is to balance rigidly and needless complexity. 
\end{frame}

\begin{frame}{}
    \lstinputlisting{reason-to-change-01.ts}
\end{frame}

\begin{frame}{}
    \lstinputlisting{reason-to-change-02.ts}
\end{frame}

\subsection{Classes should be open to extension and closed for modification.}

\begin{frame}{}
\end{frame}

\subsection{Depend on abstractions not on concrete classes.}

\begin{frame}{}
\end{frame}

\subsection{Don't call us, we'll call you.}

\begin{frame}{}
\end{frame}

\subsection{Encapsulate what varies.}

\begin{frame}{}
\end{frame}

\subsection{Favour composition over inheritance.}

\begin{frame}{}
\end{frame}

\subsection{Only talk to your friends.}

\begin{frame}{}
\end{frame}

\subsection{Program to interfaces not to implementations.}

\begin{frame}{}
\end{frame}

\subsection{Strive for loosely coupled designs among objects that interact.}

\begin{frame}{}
\end{frame}

\end{document}


